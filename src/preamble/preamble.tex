\usepackage[lmargin=1in,rmargin=1in,tmargin=1in,bmargin=1in]{geometry} % see geometry.pdf on how to lay out the page. There's lots.
\geometry{a4paper}

%%%%%% FOR TESTING ONLY %%%%%%%%%%%%%
\usepackage[english]{babel}
\usepackage{blindtext}
%%%%%%%%%%%%%%%%%%%%%%%%%%%%%%

%%%%%%%%%%%%%%%%% PACKAGES %%%%%%%%%%%%%%%%%

\usepackage{graphicx} % enhanced grpahics support
\renewcommand{\sfdefault}{cmbr} % slightly nicer than the standard sans serif font
\usepackage[onehalfspacing]{setspace} % line spacing
\usepackage[T1]{fontenc} % font encoding (fixes some font-related errors)
\usepackage{textcomp} % more font encoding
\usepackage[printonlyused]{acronym} % To handle acronyms
\usepackage{appendix} % For managing appendices (makes ToC neater)

%%%%%%%%%%%%%%%%% DOCUMENT PROPERTIES %%%%%%%%%%%%%%%%%

% Thesis details
\newcommand{\LongTitle}{My Technical Project}
\newcommand{\Me}{Joe Bloggs, Jr.}

% Number of document levels
\setcounter{secnumdepth}{4}

% Prevent hyphenation
\hyphenpenalty=5000
\tolerance=1000

%%%%%%%%%%%%%%%%% ACRONYMS AND SYMBOLS %%%%%%%%%%%%%%%%%

\newcommand{\RT}{RT\ensuremath{_{\text{60}}}}
\newcommand{\DEG}{\ensuremath{^\circ}}

%%%%%%%%%%%%%%%%% DATE %%%%%%%%%%%%%%%%%

% For placing on title page
\usepackage{datetime}
\newdateformat{MyMonth}{\monthname[\THEMONTH]}
\newdateformat{MyYear}{\THEYEAR}
\newdateformat{MyDate}{\THEYEAR}

%%%%%%%%%%%%%%%%% CAPTIONS %%%%%%%%%%%%%%%%%

\usepackage[margin=2em,singlelinecheck=on,font=sf,labelfont+=bf,labelformat=simple,labelsep=colon]{caption}

\newcommand{\captionfontsize}{\fontsize{9}{11}\selectfont}
\renewcommand\captionfont{\captionfontsize\sffamily}

%%%%%%%%%%%%%%%%% TITLES %%%%%%%%%%%%%%%%%

\usepackage[sf,raggedright,toctitles]{titlesec}

% Chapter heading properties
\titlespacing{name=\chapter}{0ex}{0ex}{.5in}[0ex]
\titlespacing{name=\chapter,numberless}{0ex}{0ex}{.5in}[0ex]

% Subsubsection heading properties
\titleformat{name=\subsubsection,numberless}[hang]{\sf}{}{0pt}{\bfseries}
\titlespacing{name=\subsubsection,numberless}{0em}{2ex}{0ex}

%%%%%%%%%%%%%%%%% TOC / LOF / LOT / LOE %%%%%%%%%%%%%%%%%

\usepackage[titles]{tocloft}

\setcounter{lofdepth}{1}

% put "Chapter #: " in ToC
\renewcommand{\cftchappresnum}{\chaptername~}
\renewcommand{\cftchapaftersnum}{:}
\newlength{\mylen} % a "scratch" length
\settowidth{\mylen}{\bfseries\chaptername\cftchapaftersnum} % extra space
\addtolength{\cftchapnumwidth}{\mylen} % add the extra space

% control indention
\setlength{\cftchapindent}{0em}
\setlength{\cftfigindent}{0em}
\setlength{\cfttabindent}{0em}

\newlength{\tocloftindent}
\setlength{\tocloftindent}{2.3em}

\setlength{\cftsecindent}{1.5em}

\makeatletter
\renewcommand{\@pnumwidth}{1.9em} % was 1.55em
\renewcommand{\@tocrmarg}{2.9em plus1fil} % raggedright toc entries (no hyphenation)
\makeatother

\setlength{\cftsubsecindent}{3.8em}

%%%%%%%%%%%%%%%%% TABLES %%%%%%%%%%%%%%%%%

\usepackage{array,color,colortbl,multirow,longtable}

\newcommand{\colheading}[1]{\multirow{2}{*}{\textbf{#1}}} % custom column heading command
\newcommand{\tablesubtitle}[2]{\multicolumn{#1}{l}{\emph{#2}}} % custom table sub heading to span the specified number of columns
\renewcommand\arraystretch{1.2} % row height

%%%%%%%%%%%%%%%%% MATHS %%%%%%%%%%%%%%%%%

\usepackage{amsmath,amsbsy,amssymb}

\allowdisplaybreaks

% define new symbols and operators
\newcommand{\R}{\mathfrak{R}} % Real operator
\newcommand{\Z}{\mathbb{Z}} % set of integers
\newcommand{\N}{\mathbb{N}} % set of natural numbers
\newcommand{\F}{\mathcal{F}} % Fourier operator
\newcommand{\ud}{\,\text{d}} % d in differential
\newcommand{\fs}{f\negthinspace{}s} % sampling frequency
\DeclareMathOperator*{\argmax}{arg\,max} % arg max
\DeclareMathOperator*{\argmin}{arg\,min} % arg min
\providecommand{\abs}[1]{\lvert#1\rvert} % abs brackets

%%%%%%%%%%%%%%%%% CITATIONS %%%%%%%%%%%%%%%%%

\usepackage[square]{natbib}

\setcitestyle{aysep={},notesep={: },citesep={;},yysep={,}}

\setlength{\bibhang}{0pt}

\newcommand{\citepos}[1]{\citeauthor{#1}'s \citeyearpar{#1}}
\newcommand{\citequote}[1]{\newline \strut\hfill \citep{#1}}
\newcommand{\cpage}[1]{p.~#1}
\newcommand{\cpages}[2]{pp.~#1--#2}

%%%%%%%%%%%%%%%%% HEADER/FOOTER %%%%%%%%%%%%%%%%%

\usepackage{fancyhdr}
\setlength{\headheight}{15pt}

\newcommand{\myfooter}{\textsf{\thepage}}

\fancypagestyle{plain}{%
\fancyhf{}%
\fancyfoot[R]{\myfooter}%
\renewcommand{\headrulewidth}{0pt}%
\renewcommand{\footrulewidth}{1pt}%
}

\pagestyle{fancy}
\fancyhf{}%
\renewcommand{\headrulewidth}{1pt} %header line width
\renewcommand{\footrulewidth}{1pt} % footer rule width
\fancyhead[R]{\textsf{\nouppercase{\leftmark}}} % header right - chapter title
\fancyfoot[R]{\myfooter} % footer right - page number

%%%%%%%%%%%%%%%%% HYPERREF %%%%%%%%%%%%%%%%%

% this should hopefully be the last entry in the preamble - defines properties of the typeset pdf file
% this package makes clickable links in the pdf
\usepackage[colorlinks,pdftex,breaklinks,pdfdisplaydoctitle,plainpages=false]{hyperref} 
\hypersetup{pdftitle={\LongTitle},pdfauthor={\Me},
    citecolor=black,%
    filecolor=black,%
    linkcolor=black,%
    urlcolor=black
}